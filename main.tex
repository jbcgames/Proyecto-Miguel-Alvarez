\documentclass{article}
\usepackage[utf8]{inputenc}
\usepackage[spanish]{babel}
\usepackage{listings}
\usepackage{graphicx}
\graphicspath{ {images/} }
\usepackage{cite}

\begin{document}

\begin{titlepage}
    \begin{center}
        \vspace*{0cm}
            
        \Huge
        \textbf{Programacion en una persona.}
            
        \vspace{0.5cm}
        \LARGE
        Pasos a seguir
            
        \vspace{1cm}
            
        \textbf{Miguel Angel Alvarez Guzman}
            
        \vfill
            
            
        \Large
        Despartamento de Ingeniería Electrónica y Telecomunicaciones\\
        Universidad de Antioquia\\
        Medellín\\
        Marzo de 2021
            
    \end{center}
\end{titlepage}

\tableofcontents
\newpage
\section{Introduccion}\label{intro}
Este proyecto tiene como fin demostrar la aplicacion de la programacion en una forma que cualquiera pueda entender, tal como se explico en la clase de el profesor Augusto Enrique Salazar Jimenez.

\section{Realizacion} \label{contenido}
Para demostrar la programacion en la vida diaria se le dara a una persona dos cartas y esta solamente leyendole unos pasos lo intentara convertir en una piramide.
\subsection{Pasos a seguir}
%
Paso°1: Juntar las dos cartas en una misma superficie.

Paso°2: Hacer coincidir sus bordes.

Paso°3: Agarrar las esquinas superiores con el pulgar y el indice

Paso°4: Apoyar la parte inferior de en una superficie plana horizontal.

Paso°5: Utilizar los otros dedos para abrir las cartas en forma de triangulo.

Paso°6: Soltar las cartas de los dedos en la posicion actual.

Paso°7: Si se cae volver a repetir desde el paso 1.

\end{document}
